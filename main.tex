\documentclass[a4paper,12pt,fleqn]{article}

\usepackage[left=2cm,right=2cm,
top=2cm,bottom=2cm,bindingoffset=0cm]{geometry}
\usepackage{xcolor}
\usepackage{hyperref}
\usepackage{cmap}					% поиск в PDF
\usepackage[T2A]{fontenc}			% кодировка
\usepackage[utf8]{inputenc}			% кодировка исходного текста

\usepackage{natbib}
\usepackage{esvect}
\usepackage{graphicx}
\usepackage{amsmath}
\usepackage{dsfont}
\usepackage{systeme}

\usepackage[russian]{babel}

\title{ВМА. Лабораторная работа 2.\\ Собственные числа и векторы матрицы}
\author{Максимов Матвей Игоревич}
\date{Декабрь 2021}

\begin{document}

\maketitle

\section{Степенной метод}
\subsection{Вариации метода}
Я реализовал три вариации степенного метода для нахождения собственных чисел и векторов матрицы: простой степенной метод для нахождения одного наибольшего собственного числа и соответствующего ему вектора, более сложную реализацию для случая, когда у матрицы два наибольших по модулю собственных значения совпадают по модулю и более простая реализация не сходится, а также еще более изощренный метод для случая, если наибольшими по модулю собственными числами является пара комплексно-сопряженных чисел. Асимптотики методов: $O(i * n^2)$ для всех алгоритмов ($i$ - количество итераций). Приведу краткое обоснование: \\
\begin{enumerate}
  \item В первой модификации на каждой итерации мы умножаем матрицу на вектор, что и дает асимптотику $O(n^2)$ на итерацию с учетом, что остальные операции над векторами производятся за линейное время.\\
  \item Во второй модификации достаточно просто в два раза больше операций, чем в первой. Асимптотика та же.\\
  \item В третьей модификации итерация также занимает $O(n^2)$ так как требует умножения матрицы на вектор, а задача минимальных квадратов хоть и решается за $O(n*m^2)$, но матрица, для которой мы решаем эту задачу, имеет всего две строки, поэтому $m$ можно принять за константу в данном случае. Но по очевидным причинам этот метод всё же имеет большую вычислительную сложность, чем второй.\\
\end{enumerate}\\
Это был самый сложный в реализации алгоритм хотя бы из-за необходимости реализации мнк и написания трех алгоритмов, а также экспериментов с алгоритмами, определяющими сходимость. Проблем в реализации было так много, что уже и не перечислить. Больше всего проблем доставил случай комплексных собственных значений.\\

\subsection{Критерий остановки}
Критерием остановки итераций я выбрал $|\lambda_{i-1} - \lambda_{i}| < \epsilon$, так как в степенном методе сходится именно последовательность $\lambda_i$. В некоторых местах также используется невязка $|A u - \lambda u|$.\\

\subsection{Выбор лучшей модификации}
Хоть модификация метода для комплексных значений и покрывает все случаи, скорость ее работы меньше скорости двух первых модификаций и, соответственно, можно взять лучшее от трех модификаций если выбрать наиболее подходящий для данной матрицы метод. Для этого я пробовал разные способы определения алгоритма:\\
\begin{itemize}
    \item Для этого я вынес итерации методов в отдельные функции и написал алгоритмы, проверяющие схождение методов. Алгоритм заключается в следующем: фиксируем количество проверочных итераций, пусть это будет $k$; далее запускаем $k/2$ итераций проверяемого метода, после чего следующие $k/2$ итераций проверяем, чтобы модуль разности $||\lambda_{i-2} - \lambda_{i - 1}| - |\lambda_{i - 1} - \lambda{i}||$ уменьшался каждую итерацию. Если это так, то запускаем проверяемую на сходимость модификацию метода уже до соответствия условию невязки, используя значения, полученные за $k$ проверочных итераций, чтобы не пересчитывать уже полученные данные еще раз. Проверку методов начинаю с самого простого с точки зрения вычислительной сложности, то есть с самой простой модификации, далее вторая модификация, а если и она не сходится, то мы имеем дело с комплексными значениями, и запускается третий метод.\\
    \item Но оказалось, что такой метод часто дает ложноотрицательное отсутствие сходимости, поэтому я попробовал другой: снова зафиксировал $k$ и теперь еще $step$ и если $|\lambda_i - \lambda_{i-step}| < \epsilon$, то метод сходится. Это тоже не дало хороших результатов. Тогда я решил для начала потестировать алгоритмы.\\
\end{itemize}

\subsection{Зависимость скорости работы от размерности матрицы}
Я провёл несколько разных тестов. Результаты меня удивили, так как судя по асимптотике (с учетом большой константы третьего метода) было вполне понятно, каких результатов ждать: первый алгоритм самый быстрый, за ним второй и третий. \\
Для начала методика тестирования. Так как степенной метод является итерационным и не на всех матрицах сходится за разумное время, встал вопрос о методике тестирования: ведь нет большого смысла тестировать алгоритм и засекать время его работы для матриц, на которых он имеет крайне низкую скорость сходимости. Я решил тестировать алгоритмы только на матрицах, на которых они сходятся за разумное количество итераций.\\
Первым тестом был тест времени работы алгоритмов и скорости их сходимости (кол-во итераций). Для этого я фиксировал размер матрицы, генерировал случайные матрицы (вся генерация в лабораторной работе осуществлялась с помощью $std::mt19937$ и $std::uniform\_int\_distribution$) и если за разумное (фиксированное) кол-во итераций метод (какой?) сходился, то я добавлял эту матрицу в список пригодных для тестирования (я генерировал по 5-10 матриц для каждой размерности для более точного определения времени работы). Процесс это небыстрый (дальше будут тесты сходимости алгоритмов), поэтому я решил использовать все возможности своего компьютера и написал процесс перебора матриц многопоточно, что конечно дало отличный множитель скорости. На этом моменте я чуть не допустил фатальную ошибку: передавал всем потокам один $seed$ для генератора. Осталось разобраться только с тем, какой алгоритм брать за эталонный (по скорости сходимости которого будем судить брать ли матрицу в список пригодных). Очевидно, если матрица сходится для первого метода, то она сойдется для всех остальных (возможно, с бОльшим кол-вом итераций, но всё же за разумное время). Как и если матрица сошлась для второго алгоритма, то она сойдется и для второго. Я провел два теста: в одном алгоритмом для выбора матрицы послужил первый, во втором - второй. \\
На графиках ниже результаты первого и второго теста. \\
\includegraphics[scale=0.6]{1.png}
\includegraphics[scale=0.6]{2.png}\\
Результаты, ожидаемо, идентичные, но без второго теста тестирование не было бы полным.\\
Третий алгоритм ожидаемо работает сравнительно долго, а вот первый и второй идут бок о бок. Я это объясняю тем, что хоть у второго алгоритма выше вычислительная сложность, он делает на почти в два раза меньше итераций, чем первый и почти в три меньше, чем третий (графики решил не вставлять, так как из-за предвыбора матриц график тут рисовать неуместно. Тенденция с разницей в количестве итераций прослеживается на всех размерностях). Теперь становится понятно, как выбирать алгоритм для того самого $auto$ режима.\\
Стоит отметить, что границы генерации чисел в этом тесте не влияют на тенденции. Прослеживается другая зависимость, но это уже следующее исследование.\\


\subsection{Общее исследование сходимости алгоритма}
Для проверки сходимости алгоритма (в том числе на разных размерностях матриц и границах генерации чисел) я провёл следующее исследование. Я построил диаграммы распределения числа сделанных итераций для трех алгоритмов для 22000 матриц\\
\includegraphics[scale=0.4]{3.png}\\
\includegraphics[scale=0.4]{4.png}\\
Тут слишком долго тестировалось, пришлось снизить количество матриц до 2200.\\
\includegraphics[scale=0.4]{6.png}\\
\includegraphics[scale=0.4]{5.png}\\
Можно сделать следующий вывод: увеличение размерности матрицы как и расширение границ генерируемых чисел в общем случае приводит к ухудшению скорости сходимости. Также последний график скорее всего демонстрирует насколько потеря точности в многочисленных расчетах в третьем алгоритме может испортить сходимость.\\


\subsection{Выбор лучшей модификации 2}
Теперь ясно, что стоит выбирать один из двух алгоритмов: быстрый $Mod 2$ или медленный $Mod 3$, который сходится для комплексных чисел. Первый алгоритм работает примерно за то же время, что и второй, обладая худшей сходимостью, так что даже не стоит тратить время на проверку сойдется ли матрица для него. Остается проверить, сходится ли второй алгоритм.\\
Я придумал просто запускать алгоритм с намного меньшей точностью и если он сошелся для этой, то предполагается, что сойдется и для исходной. Если же не сошелся для одной из них, то запускается третий метод. Это, кстати, тоже не сработало. \\
Сработал следующий алгоритм выбора: сделаем какое-то фиксированное количество итераций второго алгоритма, а затем проверим невязки $|a * \lambda - \lambda * u|$. Если невязка достаточно мала, предполагается, что алгоритм сходится. \\
Вектор $u$ (им описывается всё состояние алгоритма) можно переиспользовать, чтобы итоговый алгоритм сделал меньше итераций, я это также реализовал.\\
Приведу графики скорости работы гибридного алгоритма на матрицах, сходящихся для второго алгоритма, а также на матрицах, сходящихся для третьего (время выполнения на разных графиках сравнивать некорректно, так как использовались разные наборы для тестирования).\\
\includegraphics[scale=0.75]{7.png}
\includegraphics[scale=0.75]{8.png}\\


\subsection{Оптимизации}
Теперь, когда собрано много информации о работе алгоритмов, можно подумать об оптимизации. Так как в третьем алгоритме много времени занимает вычисление собственных значений из вектора $u$, а итераций алгоритм делает очень много, то есть смысл делать какое-то количество итераций без вычисления собственных значений и векторов. Хоть это и не асимптотическое улучшение, оно приносит заметные результаты. Я подобрал количество итераций для предварительного расчета, равное $2n$. График 9 показывает успех оптимизации:\\
\includegraphics[scale=0.75]{9.png}\\


\subsection{Результаты на матрицах из условия}
В этой задаче нужно прикрепить всего по два вектора для каждой матрицы, так что вставлю их прямо в отчет.\\
Результат на первой матрице:\\
$$\lambda = -0.916362-3.925408i; 
u = 
\begin{bmatrix}
			2.120756 -8.134934i \\
-1.000365 -2.661813i\\
-4.553012 +4.233138i\\
-8.412345 -3.359963i\\
2.463326 +3.649543i\\
-10.105070 +2.113656i\\
4.421617 +3.631416i\\
-4.185485 -1.581546i\\
-1.545673 +5.342914i\\
1.831088 -4.834847i
\end{bmatrix}
$$
$$|a * u- \lambda * u| = 0.000344$$

$$\lambda = -0.916362+3.925408i; 
u = 
\begin{bmatrix}
			
2.084882-0.053561i\\
   0.586637+0.391790i\\
  -1.279437+0.861206i\\
   0.337290+2.221788i\\
 -0.742752-0.800925i\\
  -1.080519+2.322032i\\
 -0.627932-1.272996i\\
  0.146031+1.100345i\\
  -1.377939+0.072090i\\
  1.271294-0.169695
\end{bmatrix}
$$
$$\\|a * u- \lambda * u| = 0.000085
$$
Количество итераций третьего алгоритма: 384.\\


Вторая матрица:
$$\lambda = 3.852308; 
u = 
\begin{bmatrix}
			-0.079351 \\
0.222672 \\
-0.110820 \\
-0.271102 \\
0.177505 \\
-0.251727 \\
0.188700 \\
0.000744 \\
-0.370928 \\
0.011228 \\
-0.103429 \\
-0.268585 \\
0.147884 \\
0.459585 \\
0.008074 \\
-0.382132 \\
0.003430 \\
-0.115067 \\
-0.315165 \\
-0.124345
\end{bmatrix}
$$
$$|a * u- \lambda * u| = 0.000207$$
$$|u| = 0.999954$$


$$\lambda = -3.852308; 
u = 
\begin{bmatrix}
0.000277\\ 
0.000148\\ 
-0.000214\\ 
0.000233\\ 
-0.000306\\ 
-0.000061\\ 
0.000337\\ 
0.000188\\ 
-0.000024\\ 
0.000263\\ 
0.000136\\ 
-0.000339\\ 
0.000398\\ 
-0.000256\\ 
0.000135\\ 
-0.000081\\ 
-0.000361\\ 
-0.000416\\ 
0.000235\\ 
-0.000115
\end{bmatrix}
$$
$$|a * u- \lambda * u| = 0.000207$$
$$|u| = 0.001126$$
Количество итераций второго алгоритма: 79.\\
Тут можно заметить, что другие алгоритмы не находят второй корень, а этот находит, да еще и с маленькой невязкой. Как так? По определению собственного вектора он должен быть ненулевым, а вот тут алгоритм возвращает почти что нулевой вектор. Но так как его норма всё же достаточно велика по сравнению с $\epsilon = 10^{-6}$, алгоритм засчитывает этот вектор как валидный. Следовательно, можно сделать вывод о плохой обусловленности задачи нахождения собственных векторов для данной матрицы вторым алгоритмом.


\section{Метод Данилевсвого и нахождение корней многочлена}
\subsection{Метод Данилевского. Общие сведения}
Метод достаточно прост: приводим матрицу к форме Фробениуса ортогональными преобразованиями в точности по конспекту Фалейчика (ясное дело с выбором по строке и столбцу). Также учитываем вырожденный случай (тоже описан в конспекте). В таком случае можно не запускать метод от подматрицы, а просто пропустить итерацию. В итоге получится матрица, состоящая из матриц Фробениуса на диагонали. Далее при реализации самого метода Данилевского не забудем учесть это, аккуратно разобрав матрицу на такие подматрицы (без их копирования, а также использования рекурсии). Также нигде не будем забывать учитывать, что метод много где гарантирует нулевые элементы и не будем лишний раз их трогать.

\subsection{Преобразование к форме Фробениуса}
Асимптотика алгоритма очевидно равна $O(n^3)$, так как само преобразование в форму Фробениуса немногим отличается от обычного метода Гаусса. Больше тут сказать нечего: реально метод Гаусса, только чуть сложнее.\\
График скорости работы алгоритма на разных размерностях:\\
\includegraphics[scale=0.75]{10.png}\\
Ожидаемый $n^3$.

\subsection{Метод Данилевского}
Метод Данилевского и вовсе работает за линию, так как ему нужно только посмотреть диагональ матрицы для нахождения интересующих нас подматриц, а затем для каждой такой подматрицы извлечь многочлен из ее первой строки. Ясно, что порядок многочлена будет в точности равен размерности матрицы, следовательно, при аккуратной реализации, это будет работать за $O(n)$. Тут важно заметить, что моя реализация метода Данилевского не выдает конкретный многочлен как ответ, если матрица является вырожденной. Метод выдает столько многочленов, сколько матриц Фробениуса находится на диагонали обрабатываемой матрицы, а искомый многочлен - это их произведение. Так как в дальнейшем нам придется раскладывать многочлен (находить корни), то перемножать многочлен было бы глупым решением. Если всё же нужно получить многочлен, у меня есть функция, которая умножает массив многочленов и выдает один многочлен - их произведение. \\
График скорости работы алгоритма:\\
\includegraphics[scale=0.75]{11.png}\\
Извиняюсь за обрезанный текст, $matplotlib$ почему-то решил, что тут он именно так его поместит.

\subsection{Нахождение корней многочлена. Метод Ньютона и бисекции}
Итак, у нас есть многочлен и нам нужно найти его корни. Для начала я реализовал метод, который основывается на том, что можно точно посчитать производную многочлена. В такому случае можно найти экстремумы многочлена как корни производной (ее корни находим так же - рекурсивно), а искомые корни многочлена будут лежать между экстремумами, будем находить их бисекцией. Нужно не забыть сокращать многочлен на порядок производной чтобы коэффициенты не становились слишком большими. Звучит отлично, но на практике я не был впечатлен результатами. Я придумал другой метод: будем запускать метод Ньютона от какой-то очень дальней точки, он будет сходится к какому-то корню, который мы будем уточнять методом бисекции. Производную для метода Ньютона посчитаем аналитически, так как это будет точнее, чем считать ее численно (еще и быстрее). Затем поделим многочлен на $(x - root)$ и будем повторять это пока не дойдем до многочлена первой степени. Достаточно большую границу будем брать по формуле:
$$\max(\frac{|a_n|}{b + |a_n|}, 1 + \frac{c}{|a_0|})$$
$$b = \max(|a_0|, |a_1|, ..., |a_{n - 1}|)$$
$$c = \max(|a_1|, |a_2|, ..., |a_n|)$$

\subsection{Нахождение собственных векторов}
Во-первых, зная собственное число, можно рассмотреть слау $(A - \lambda E)*u = 0$. Так как $\lambda$ - собственное число матрицы, то матрица $(A - \lambda E)$ будет иметь нулевой определитель, следовательно однородная слау будет иметь ненулевое решение - искомый вектор. Такое решение будет находить каждый собственный вектор за $O(n^3)$, что не очень-то хорошо с учётом того, что таких векторов тоже может быть $n$. Я думал над каким-либо треугольным разложением для такой матрицы, но $\lambda E$ всё портит.\\
Также я нашел способ быстрее. Можно заметить, что для $F - \lambda E$ ($F$ - матрица формы Фробениуса) собственный вектор для значения $\lambda$ считается как степени $\lambda$ начиная с правой стороны (легко показать, если последовательно решить СЛАУ, то есть занулить всю первую строку, а затем положить последний элемент искомого вектора равным единице). Тогда остается только привести его к вектору изначальной матрицы. Для этого запомним преобразования столбцов, которые мы делали для приведения матрицы к форме Фробениуса и выполним их для этого вектора в обратном порядке (с оговоркой, что при прибавлении столбца $a$ к $b$ будем наоборот прибавлять $b$ к $a$). Количество преобразований, очевидно, ограничено $2n + n^2$, тогда асимптотика алгоритма будет $O(s * n^2)$ ($s$ - количество собственных значений).

\subsection{Матрицы из условия}
Чтобы не загромождать отчет я прикреплю решения отдельным файлом. Тут я находил собственные векторы вторым способом.


\section{QR Алгоритм}
Снова-таки алгоритм написан ровно по Фалейчику: приведение к форме Хессенберга, а затем итерации вращения. Асимптотика алгоритма: $O(n^3 + i * n^2)$. Критерий остановки такой: делаем итерации пока на диагонали есть пересекающиеся квадраты, либо количество нулей под главной диагональю $< \frac{n - 1}{2}$. Если оба этих критерия перестали выполняться, то итерации можно останавливать. Далее извлекаем собственные значения, тут совсем просто. Можно найти собственные векторы способом, описанным выше, за $O(n^4)$. Или же привести к форме Фробениуса и найти за $O(n^3)$.\\
Из проблем в ходе реализации можно перечислить все деления в алгоритме: и в отражениях, и в поворотах. Во всех местах, где есть деление, стоит отдельно рассматривать случай нулевого знаменателя (обычно это означает, что нужный нам элемент(ы) уже нулевой).\\
График времени работы алгоритма (методика тестирования такая же как в задаче 1).
\includegraphics[scale=0.75]{12.png}\\

\subsection{Сходимость}
Снова-таки проведем такие же тесты как для задачи 1.\\
\includegraphics[scale=0.5]{13.png}\\
\includegraphics[scale=0.5]{14.png}\\
График красивый, а ситуация страшная:\\
\includegraphics[scale=0.5]{15.png}\\
Можно сделать интересный вывод о том, что QR-алгоритм работает лучше степенного метода на матрицах с большими по модулю числами, но хуже для матриц больших размерностей.

\subsection{Матрицы из условия}
Чтобы не загромождать отчет я прикреплю решения отдельным файлом. Для демонстрации первого алгоритма нахождения собственных векторов, я нашел соответствующие векторы им.

\end{document}
